%
% File acl2016.tex
%
%% Based on the style files for ACL-2015, with some improvements
%%  taken from the NAACL-2016 style
%% Based on the style files for ACL-2014, which were, in turn,
%% Based on the style files for ACL-2013, which were, in turn,
%% Based on the style files for ACL-2012, which were, in turn,
%% based on the style files for ACL-2011, which were, in turn, 
%% based on the style files for ACL-2010, which were, in turn, 
%% based on the style files for ACL-IJCNLP-2009, which were, in turn,
%% based on the style files for EACL-2009 and IJCNLP-2008...

%% Based on the style files for EACL 2006 by 
%%e.agirre@ehu.es or Sergi.Balari@uab.es
%% and that of ACL 08 by Joakim Nivre and Noah Smith

\documentclass[11pt]{article}
\usepackage{acl2016}
\usepackage{times}
\usepackage{url}
\usepackage{latexsym}

\aclfinalcopy % Uncomment this line for the final submission
%\def\aclpaperid{***} %  Enter the acl Paper ID here

%\setlength\titlebox{5cm}
% You can expand the titlebox if you need extra space
% to show all the authors. Please do not make the titlebox
% smaller than 5cm (the original size); we will check this
% in the camera-ready version and ask you to change it back.

\newcommand\BibTeX{B{\sc ib}\TeX}

\title{INFO 6300 Project : Sentiment and Belief detection using LSTMs}

\author{First Author \\
  Anusha Chowdhury \\
  MS, Computer Science \\
  {\tt ac2633@cornell.edu} 
  %\\\And
  %Second Author \\
  %Affiliation / Address line 1 \\
  %Affiliation / Address line 2 \\
  %Affiliation / Address line 3 \\
  %{\tt email@domain} \\}
}
\date{May 1, 2017}

\begin{document}
\maketitle
\begin{abstract}
In this project, we try to solve the BeSt Evaluation task 2016, and the goal is to use  
Long Short Term Memory networks(LSTMs) for sentiment and belief detection given pairs of words
from a document. In particular we look at the sentiment polarity, that is whether it is none,
positive or negative sentiment. We use tensorflow 1.0.0 for implementation of the model
and then evaluate the model using the standard precision, recall and F1 scores.
\end{abstract}

\section{Introduction}
The goal of this project is to solve the BeSt Evaluation task from TAC(Text Analysis Conference 2016).
\subsection{Task definition} 
The 2016 BeSt Evaluation is an evaluation of sentiment and belief detection with source and target, 
where sources are named entities and targets are named entities or events or relations.
The evaluation is interested in sources, attitudes, and targets like which entity has what mental attitude towards
another entity. The source is an entity of type Person, Geo­Political Entity (GPE), or Organization.  The
target can be any relation, or any event.  In addition, for sentiment only, the target can
also be any entity. 
The evaluation includes belief and sentiment. We will look at the subproblem that given a pair of words, one has 
to predict whether the polarity is positive, negative or none for sentiment. Once we experiment with the model for this
and achieve a good accuracy, we can extend the model to belief as well in a similar manner.
\subsection{The Dataset} 
The documents consist of blog posts, newswires and quotes(labelled as dfweird in the attached folder).
Gold entities, relations, and events (EREs) and predicted EREs are provided in the dataset.
\section{Background and Related Work}
LSTM-RNNs  (Hochreiter  and  Schmidhuber, 1997)  have  been  applied  to  many  
sequential   modeling   and   prediction   tasks,   such as  machine  translation  
(Bahdanau  et  al.,  2014),    speech   recognition (Graves  et  al.,  2013) and  
named entity recognition (Hammerton,  2003).
To achieve fine-grained   opinion   extraction, researchers have focused on extracting subjective 
phrases using a CRF-based approach from open-domain  text  such  as  news  articles(Yang and Cardie, 2012). 
In existing literature, LSTMs have been used for joint extraction of opinion entities 
and relations(Arzoo and Claire, 2016).

\section{Experiments}
\subsection{Tensorflow as a beginner : MNIST dataset}
TensorFlow is a powerful library for doing large-scale numerical computation. 
One of the tasks at which it excels is implementing and training deep neural networks.
Being a beginner in neural networks and their implementations in tensorflow, I decided to first do an
experiment on a subset of the MNIST digit classification dataset(attached).

We referred to the deep MNIST website (MNIST,16) for the tensorflow implementations in python and then decided to make 
changes to the input and output as required. So we chose a subset of the MNIST dataset consisting of 4000 images, each
of which has 784 features(28X28 pixels) for training the model and then tested it on a test dataset of 800 images.
We used Tensorflow because it gives fast implementations of neural networks. We built a multilayer convolutional 
neural network to train and test on this data. We use weights(W) and biases(b) 
to suit the shapes of these tensors, that is in accordance with the image sizes(784 is to match the input 
feature vector) and 10 is the number of categories(digits can be 0 to 9). We used argmax method of tensorflow to find the 
index which had the highest value of probability(hence being the most probable digit) after doing softmax. 
We also used dropout to prevent overfitting the model. 
We experimented with 20000, 30000 and 40000 iteration sizes(please refer to the attached code for details) and 
batch sizes of 50, 70, 100 and 110. Out of these, the one with 30000 iteration size and batch size of 70 gave around 
98.5$\%$ accuracy. With 40000 iteration size and 110 batch size I got 99.25$\%$ accuracy.
So from this I learnt implementations of neural networks in tensorflow, the fact that batch size and epochs matter
as parameters and that there is a sweet spot after which error may increase due to overfitting.


\section{The Model}
\subsection{Hyperparameters}
The different hyperparameters that we had to specify for this model were 
maximum epoch, batch size, hidden size, number of layers, learning rate, training fraction, dropout, maximum sequence length 
and steps per checkpoint(please refer to config.ini in the code folder).
Among these we kept the hidden size always as 50, number of layers as 2, learning rate as 0.01, batch size as 1,
steps per checkpoint as 50 and maximum sequence length as 5000. 
For the rest of the parameters we experimented with different values. Actually we should be 
experimenting with each of these parameters, but as we have too many, due to time constraints we decided to 
restrict ourselves to a few of them.


\section*{Acknowledgements}
I would like to thank my advisor Professor Claire Cardie for guiding me throughout this project.
Without her help, this project would not have been possible. I would also like to thank her for 
all the advising office hours, which were very helpful to ensure whether I was on the right track.
I would like to give special thanks to Arzoo Katiyar for helping me with the tensorflow implementations, coding and
improving the model. Also her existing work(Arzoo and Claire,2016) was extremely helpful since it was very much
related with this project. I had regular meetings with her and that helped me a lot to progress fast.
I would also like to thank Vlad and Esin for helping me with collection of the dataset and related materials.

%\bibliography{acl2016}
%\bibliographystyle{acl2016}

\end{document}
